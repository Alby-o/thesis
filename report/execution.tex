% !TeX root = ./report.tex
\chapter{Execution}
Experimentation began with the CompCert compiler and the provided assembly annotation tools, outlined in section \ref{sec:compCert}. It was found that the CompCert compiler could not handle all cases necessary for the wpif analysis, specifically volatile variables. As a result, the testing moved on to other techniques. Following this, the GNU C extension for inline assembly was explored as a possibility to preserve annotations in C in section \ref{sec:inlineAssembly}. This technique prevailed and was found to be excellent in handling assembly annotations by injecting comments in to the compiled assembly output. This technique was enhanced by developing a python program to inject inline assembly into the source C files to allow for enhanced analysis and furthermore avoids restricting the program to GNU extension supporting compilers. As a result of the success, modifying the compiler was not explored due to success documented in other research such as the work conducted by Vu et al. \cite{vu2020secure}. This allowed for further development and improvement of the inline assembly method.

\section{CompCert AIS Annotations}
\label{sec:compCert}
CompCert is unfortunately not a free tool, however, for research purposes it can be used freely. The specifications of the CompCert install can be seen in Table \ref{tab:compcertInstall}. 

\begin{table}
    \begin{center}
        \begin{tabularx}{\linewidth} { 
            | >{\hsize=.7\hsize\linewidth=\hsize}X 
            | >{\hsize=1.3\hsize\linewidth=\hsize}X  | }
            \hline
            OS Name & Ubuntu 20.04.2 LTS \\
            \hline
            OS Type & 64-bit \\
            \hline
            Processor & Intel® Core™ i7-6700K CPU @ 4.00GHz × 8 \\
            \hline
            Instruction Set & x86-64 \\
            \hline
            \raggedright
            CompCert Version & The CompCert C verified compiler, version 3.7 \\
            \hline
        \end{tabularx}
    \end{center}
    \caption{CompCert install specifications}
    \label{tab:compcertInstall}
\end{table}

Testing was initially conducted using the \textit{comment.c} test file. The goal is to propagate the comment down to assembly where it can be used and interpreted. To do so, the comment in the source code needs to be replaced with a call to generate an annotation in the compiled assembly. Fortunately, with the CompCert compiler, this functionality is builtin. This assembly annotation is created through the use of the \code{\_\_builtin\_ais\_annot} function described in \ref{sec:relatedWorkCompCert}. The following builtin annotation was placed in line 2, within the main function in \textit{comment.c}.

\lstinputlisting[firstnumber=2, firstline=2, lastline=2, caption=comment.c]{source/compCert/comment.c}

Within this annotation, \code{\%here} is used to represent the location within the program. If the location is not important, \code{\%here} can be omitted. The comment, \code{"Critical Comment"}, has been included to represent some kind of critical comment that is required to conduct a static analysis on the output. To compile the source to assembler only the following command was used:

\begin{lstlisting}[numbers=none]
$ ccomp comment.c -O0 -S
\end{lstlisting}

Here -O0 is used to specify to perform no optimisations during compilation. The full compiled output can be seen in Appendix \ref{app:compCertAnnotatedAssembly}. Below is a snippet of the compiled assembly.

\lstinputlisting[firstnumber=16, firstline=16, lastline=24, caption=comment-O0.s]{source/compCert/annotated/comment-O0.s}

The annotation is stored within assembler directives. Assembler directives are not a part of the processor instruction set, however, are a part of the assembler syntax. Assembler directives all start a period (.). On line 19 a new section has been created, named \code{"\_\_compcert\_ais\_annotations"}. Following the declaration of the section is an ascii string, locating the source of the annotation within the source program \textit{comment.c}. Line 23 provides the comment we aimed to preserve with our annotation. Thus, CompCert has shown an initial success in preserving annotations in the form of comments.

Additionally, one major benefit of compCert annotations is that they do not modify the source program, as they are inserted at the end of the program as an assembler directive metadata.

When experimenting with annotated variables, the first issues began to arise. The test file \textit{variable.c} contains several variables with their types to preserve to assembly. The annotations behaved as expected for the types:

\begin{itemize}
    \item int,
    \item char,
    \item short, 
    \item long, and
    \item any signed or unsigned variations of the above mentioned types.
\end{itemize}

However, the CompCert annotations does not support floating point types. Upon compiling \textit{variable.c} the following errors were generated.

\begin{lstlisting}[numbers=none]
variable.c:13: error: floating point types for parameter '%e1' are not supported in ais annotations
variable.c:15: error: floating point types for parameter '%e1' are not supported in ais annotations
2 errors detected.
\end{lstlisting}

This result shows that it is impossible to use the CompCert embedded program annotations for floating point types, vastly restricting its potential use as a technique for a wpif analysis.

It was discovered soon after that the CompCert annotations are unable to handle volatile variables, generating the follow error upon compiling \textit{volatile.c}.

\lstinputlisting[numbers=none]{source/compCert/annotated/volatile-O0.s}

Unfortunately, this result shows that the CompCert AIS annotations approach is not suitable for wpif analysis. The wpif analysis requires use of volatile variables. This is because the primary purpose of the wpif technique is to verify security policy across concurrent programs. Shared variables within concurrent programs can change at any time, and as such it is imperative that shared variables are marked as volatile. As the CompCert AIS annotations cannot handle volatile variables, annotations required for wpif analysis cannot be generated.

Aside from the aforementioned issues, the CompCert AIS annotations performed excellently in generating annotations. The location of global variables in memory are easily identified, as shown in \textit{rooster.c}. The CompCert AIS annotations must be placed within a method and called as if it was its own function. This creates some confusion when dealing with global variables. However, placing annotations on global variables at the start of main is a perfectly valid method of preserving these annotations. As the location of the annotation within the program is no longer important, the \code{\%here} format specifier can be omitted.

\lstinputlisting[firstline=84, firstnumber=84, caption=rooster-O0.s]{source/compCert/annotated/rooster-O0.s}

From \textit{rooster.c}, the comment "CRITICAL COMMENT" has been annotated from lines 88 to 91, and the comment "EXCEPTIONAL" has been annotated from lines 99 to 102. Most notably, the global variable \code{goose} has been annotated from lines 92 to 98. Reconstructed, the string \code{"L(mem(goose, 4)) = medium"} has been preserved. Thus, the CompCert annotations can successfully preserve annotations on global variables.


Another interesting problem faced when working with CompCert AIS annotations is found when working with structs. If the programmer wants to annotate a member of a struct for all structs of that type, each instance of that type of struct must be annotated when using CompCert AIS annotations. This is because CompCert treats \code{\_\_builtin\_ais\_annot()} as a call to an external function. As such, an annotation cannot be created from outside a method, similar to when dealing with global variables. An example of this process can be seen in \textit{password.c}. Within the program, each instantiation of the struct \code{user\_t} requires another annotation. 

\lstinputlisting[firstline=17, firstnumber=17, lastline=33, caption=password.c, language=C]{source/compCert/password.c}

The compiled output is as expected, with an annotation within the assembly for each of the annotations created within the source file.

\lstinputlisting[firstline=320, firstnumber=320, caption=password-O0.s]{source/compCert/annotated/password-O0.s}

As seen in the assembly annotations, the location of the struct members have been preserved. Line 324 contains the annotation \code{L((reg("rbp") + 264)) = high}. This annotation notifies that the variable stored in register rbp with an offset of 264 has a security classification of high. Thus, another success for CompCert AIS annotations.

\subsection{Quality Analysis}
To complete a quality analysis, a comparison of the assembly will need to be conducted with and without annotations. The CompCert assembly output can be seen in Appendix \ref{app:compCertAssembly}. As we are primarily concerned with the annotated assembly after aggressive optimisations have been performed, the assembly with optimisation level 3 will be compared. To begin with, the assembly for \textit{comment.c} will be compared. Performing a diff on the annotated and non annotated assembly produces the following diff:

\lstinputlisting[caption=comment-O3.s diff]{source/compCert/diff/comment-O3.diff}

The diff explains some interesting differences in the assembly. To begin with, an additional label \code{.L100:} has been inserted. The only other notable difference is in the compcert annotations. The reason behind the additional label is to allow the location of the annotation to be identified, as can be seen in line 11 of the diff. Thus, this shows a success. There is no difference in the compiled output, even with aggressive compiler optimisations turned on. 

Next, \textit{variable-O3.s} will be compared.

\lstinputlisting[caption=variable-O3.s diff]{source/compCert/diff/variable-O3.diff}

Similar to before, additional labels \code{.L100:} to \code{.L104:} have been inserted to identify the annotations from within the source code. However, with aggressive optimisations turned on, an interesting change has occurred to the annotations. As the variables have been optimised out of the compiled assembly, the annotations no longer make sense. Line 16 of the diff shows the annotation for variable \code{a} from \textit{comment.c}. However, as the variable has been completely optimised out, rather than the location of the register being preserved in the annotation, only the value stored within \code{a} has been preserved. In this case, that value was \code{-10}.

Although the annotations do not interfere with the compiler optimisations, the compiler optimisations have rendered the annotations useless. Unfortunately, in cases such as these, there is not much to be done. 

Another interesting case arises when a loop is optimised out by the compiler. Take for example \textit{count.c}

\lstinputlisting[caption=count.c, language=C]{source/compCert/count.c}

As count will always be zero, the loop will be optimised out when optimisations are turned on. As CompCert treats \code{\_\_builtin\_ais\_annot()} as a call to an external function, it too will be optimised out with aggressive compiler optimisations. Let's compare the assembly with and without compiler optimisations. 

\lstinputlisting[caption=count-O0.s]{source/compCert/annotated/count-O0.s}

With optimisations turned off, the annotation is preserved, as seen from lines 27-31. Following the annotation, the location of the annotation can be located with label \code{.L102}. However, with compiler optimisations, it would be expected that \code{.L100} and \code{.L102} will be optimised out. This can be seen in \textit{count-O3.s}.

\lstinputlisting[caption=count-O3.s]{source/compCert/annotated/count-O3.s}

As expected, the optimised assembly has completely removed the annotation. In the case of a wpif analysis, this is not a large concern. Although preserving loop invariants is necessary, if the loop is optimised out by the compiler it is no longer of concern and the annotations are no longer necessary.


\section{CompCert Builtin Annotations}

Upon revisiting CompCert AIS annotations, it became apparent that the form of annotations being experimented on previously, namely AIS annotations, were not the only form of C annotations available to CompCert. The CompCert AIS annotations are built on top of the CompCert builtin annotations. The AIS annotations are built primarily for worst case execution time analysis, and thus it was assumed that these annotations would be the most suitable for wpif purposes. However,  Similar to the AIS annotations, the CompCert builtin annotations can be called upon using a method call, \code{\_\_builtin\_annot()}. To begin with, annotating \code{"CRITICIAL COMMENT"} will be tested. The call to \code{\_\_builtin\_annot()} can be seen below.

\lstinputlisting[firstnumber=2, firstline=2, lastline=2, caption=comment.c]{source/compCertBuiltin/comment.c}

Once compiled to assembly, rather than outputting to the bottom of the assembly within assembler directives. Instead, it is placed in the middle of the assembly as a comment. The annotated assembly from the builtin annotation can be seen below.

\lstinputlisting[caption=comment-O0.s]{source/compCertBuiltin/annotated/comment-O0.s}

On line 12, the annotation has been inserted, with "Critical Comment" preserved. Rather than using the \code{\%here} directive to locate the annotation, the annotation itself has been placed directly within source code in the relevant location. A variable annotation is created in a similar fashion to the CompCert AIS annotations. 

\lstinputlisting[firstnumber=3, firstline=3, lastline=3, caption=variable.c, label=lst:compCertBuiltinVariable]{source/compCertBuiltin/variable.c}

Within Listing \ref{lst:compCertBuiltinVariable}, the variable a has been annotated with its type. Rather than using \code{\%e1} and so on to reference each element, positional parameters \code{\%1}, \code{\%2}, etc, can be used to locate variables. The annotated assembly is as expected.

\lstinputlisting[firstnumber=6, firstline=6, lastline=28, caption=variable-O0.s]{source/compCertBuiltin/annotated/variable-O0.s}

Significantly, there is no issues caused by annotating floating point numbers, a surprising success. This suggests that CompCert builtin annotations may able to perform annotations that CompCert AIS annotations cannot. Also worth noting is how this method performs under compiler optimisations. Similar to AIS annotations, when variables are optimised away only their value remains within the annotation.

\lstinputlisting[firstnumber=6, firstline=6, caption=variable-O3.s]{source/compCertBuiltin/annotated/variable-O3.s}

The primary reason AIS annotations were not appropriate as a method for wpif analysis was because the technique could not handle volatile variables. However, the builtin annotations can. Below is the builtin annotation for a volatile global variable \code{x}.

\lstinputlisting[caption=volatile.c]{source/compCertBuiltin/volatile.c}

Successfully, the annotation is compiled into the assembly. The annotated assembly can be seen in Listing \ref{lst:compCertAnnotatedVolatile}.

\lstinputlisting[firstnumber=7, firstline=7, caption=volatile-O3.s, label=lst:compCertAnnotatedVolatile]{source/compCertBuiltin/annotated/volatile-O3.s}

Further testing shows the CompCert builtin annotation method is viable for all purposes for wpif analysis.

\subsection{Quality Analysis}

As the AIS annotations are built on top of the functionality of the CompCert builtin annotations, the results of the quality analysis are very similar. For the majority of annotations observed, the only difference between the annotated and non-annotated assembly was the annotations. Take for example \textit{variable-O0.s}.

\lstinputlisting[lastline=28, caption=variable-O0.s diff]{source/compCertBuiltin/diff/variable-O0.diff}

Although it appears as if many changes have been made, the only modifications are to the locations within memory and registers. The last half of the diff has not been included here for spacial purposes.

However, interesting changes occur when working with more complex programs. For example, \textit{volatile-O3.s}.

\lstinputlisting[caption=volatile-O3.s diff]{source/compCertBuiltin/diff/volatile-O3.diff}

Line 6 of the diff shows and interesting occurrence. An additional move statement has been added. This was behaviour not observed from the CompCert AIS annotations. Testing of programs such as \textit{pread.c} have shown the same results, with extraneous statements inserted. The cause of these additional statements are unknown. However, it is likely the reason AIS annotations did not support volatile variables was because handling compiler optimisations with volatile variables is quite difficult. It is likely the case that because volatile variables have been annotated, the annotations could not be created without reverting some compiler optimisations. 

\section{Inline Assembly}
\label{sec:inlineAssembly}

Extended \code{asm} is a GNU Extension supported by many compilers. As such, it presents itself as an excellent method for preserving annotations to assembly. As it allows programmers to write assembly code within the C program, it provides an opportunity to hook in to this functionality and utilise it for annotation purposes. To begin with, a very simple program will be experimented on to find the limits of inline assembly and to assess if it is fit for wpif analysis purposes. The compiler used was clang, with the install specifications available in Table \ref{tab:clangInstall}. To begin testing, \textit{comment.c} will be used.

\begin{table}
    \begin{center}
        \begin{tabularx}{\linewidth} { 
            | >{\hsize=.7\hsize\linewidth=\hsize}X 
            | >{\hsize=1.3\hsize\linewidth=\hsize}X  | }
            \hline
            OS Name & Ubuntu 20.04.2 LTS \\
            \hline
            OS Type & 64-bit \\
            \hline
            Processor & Intel® Core™ i7-6700K CPU @ 4.00GHz × 8 \\
            \hline
            Target & x86\_64-pc-linux-gnu \\
            \hline
            \raggedright
            Compiler & clang version 10.0.0-4ubuntu1 \\
            \hline
        \end{tabularx}
    \end{center}
    \caption{Clang install specifications}
    \label{tab:clangInstall}
\end{table}

Within the assembly, an inline comment can be created by inserting a \code{\#} at the beginning of the line or comment. The goal with this test is to try and preserve the annotation "CRITICAL COMMENT" to assembly by inserting it as a comment within the assembly. The following line was inserted within the main method of \textit{comment.c}. All the source files used for the inline assembly method can be seen in Appendix \ref{app:InlineC}.

\lstinputlisting[firstnumber=2, firstline=2, lastline=2, caption=comment.c, language=C]{source/inline/comment.c}

Here, the call to insert inline assembly is treated as a call to a method, similar to CompCert. The first argument takes the \code{AssemblerTemplate}. The \code{AssemblerTemplate} is the template used for formatting the input operands, output operands and the goto parameters. In this case only a comment is inserted in assembly, and as such the input operands and output operands parameters are omitted. The full compiled output can be seen in Appendix \ref{app:inlineAnnotatedAssembly}.

\lstinputlisting[firstnumber=6, firstline=6, lastline=21, caption=comment-O0.s]{source/inline/annotated/comment-O0.s}

Lines 15 to 17 show the annotation preserved within the assembly. As can be seen, the compiler does not understand the extent of the assembly provided to it. As such, it treats it as a 'black box', and inserts it where relevant within the assembly instructions. In comparison to CompCert its placement within the assembly is inconvenient. As it is not neatly packaged at the bottom of the assembly it must be parsed out. To distinguish assembly comments from important annotations, the string \code{"annotation: "} will be inserted before any annotations to allow for easier parsing. Thus, the annotation described from \textit{comment.c} would be inserted as:

\begin{lstlisting}[firstnumber=2, language=C]
    asm ("# annotation: CRITICAL COMMENT");
\end{lstlisting}

Following, the compiled output would be:
\begin{lstlisting}[firstnumber=15]
  #APP
  # annotation: CRITICAL COMMENT
  #NO_APP
\end{lstlisting}

One may suggest using the \code{APP} and \code{NO\_APP} comments above and below our annotation to identify it, however, these comments are system and compiler dependent. As such, a more robust solution has been used here.

Following this success, annotating the location of a variable is the next challenge. With extended \code{asm} input and output operands are available to allow programmers to work with variables. The goal is to use these mechanisms to identify the location of variables alongside their annotated data. To begin with, the location of a simple integer will be attempted to be preserved. The format of the extended asm is as follows:

\begin{lstlisting}[language=C, numbers=none]
asm asm-qualifiers ( AssemblerTemplate 
    : OutputOperands
    : InputOperands
    : Clobbers
    : GotoLabels)
\end{lstlisting}

Where \code{asm-qualifiers}, \code{OutputOperands}, \code{InputOperands}, \code{Clobbers} and \code{GotoLabels} are optional parameters. In this experimentation, asm-qualifiers will not be used as any inline assembly generated will not modify the value of any variables nor jump to any labels.

Previously the \code{AssemblerTemplate} was used as a string. However, the string can be templated to locate the value of a variable. Take the following example:

\begin{lstlisting}[language=C, numbers=none]
asm("# annotation: %0 = int" : "=m"(a))
\end{lstlisting}

Here \code{\%0} is used to refer to the first output operand, a. Output constraints must begin with either a \code{'='} or a \code{'+'}. A constraint beginning with a \code{'='} is for variable overwriting an existing value, whereas a \code{'+'} is used when reading and writing. Constraints are used to specify what operands are permitted. In this case, \code{'m'} is used, signifying a memory operand is allowed, with any kind of address the machine supports. Compiling the inline assembly with no optimisations creates the following annotation.

\begin{lstlisting}[firstnumber=30, caption=variable-O0.s]
#APP
# annotation: -20(%rbp) = int
#NO_APP
\end{lstlisting}

This annotation was created without modifying any additional assembly instructions, seemingly a success. However, when optimisations are turned on an interesting result occurs.

\begin{lstlisting}[firstnumber=6, caption=variable-O3.s, label=lst:inlineVariableOutputOperand]
main:                                   # @main
	.cfi_startproc
# %bb.0:
                                        # kill: def $edi killed $edi def $rdi
	movl	$-10, -4(%rsp)
	#APP
	# annotation: -4(%rsp) = int
	#NO_APP
	movl	-4(%rsp), %eax
	addl	%edi, %eax
	addl	$134217635, %eax        # imm = 0x7FFFFA3
	retq
\end{lstlisting}

Although the annotation is preserved to assembly, a number of optimisations performed by the compiler have been reverted to allow the location of the variable \code{a} to be preserved. This is because the compiler does not understand what the inline assembly does outside of what information has been provided to it. In this case, the compiler was provided inline assembly that used some kind of memory operand with the intention to overwrite it. Thus, the compiler was required to remove optimisations to allow the the rewritten value of \code{a} to be propagated through the program successfully. 

This appears like a failure of the inline assembly method, however, constructing the statement differently should allow for less modification of the assembly by the compiler. Rather than instructing the compiler that the assembly overwrites the existing value, \code{'+'} can be used to instruct it that our assembly reads and writes. The following statement is inserted in the C source.

\begin{lstlisting}[language=C, numbers=none]
asm("# annotation: %0 = int" : "+m"(a))
\end{lstlisting}

However, this results in the same outcome as Listing \ref{lst:inlineVariableOutputOperand}. The issue occurring here is caused due to the output operand. As the inline \code{asm} is notifying the compiler that the value is modified, it removes optimisations to allow this. Instead, using an input operand may allow the optimisation to run, as instead the compiler has been notified that only the value is being read. The following \code{asm} is inserted in \textit{variable.c}.

\begin{lstlisting}[language=C, numbers=none]
asm("# annotation: %0 = int" : : "m"(a))
\end{lstlisting}

The \code{'+'} constraint has been removed from the operand as it no longer applies to an input operand. Listing \ref{lst:inlineVariableInputOperandO0} shows the assembly with no optimisations.

\begin{lstlisting}[firstnumber=23, caption=variable-O0.s, label=lst:inlineVariableInputOperandO0]
	movl	$0, -4(%rbp)
	movl	%edi, -8(%rbp)
	movq	%rsi, -16(%rbp)
	movl	$-10, -20(%rbp)
	#APP
	# annotation: -20(%rbp) = int
	#NO_APP
	movsd	.LCPI0_0(%rip), %xmm0   # xmm0 = mem[0],zero
	movss	.LCPI0_1(%rip), %xmm1   # xmm1 = mem[0],zero,zero,zero
	xorl	%eax, %eax
\end{lstlisting}

Here, the annotation can be seen in line 28, with the location of the variable successfully identified. However, interestingly the instructions from lines 30-32 were previously above the instructions from lines 23-26. Although this makes no difference to the runtime of the program, it is an interesting change caused by the inline \code{asm}. 

Additionally, the optimised assembly using the input operands has the same issue as before from Listing \ref{lst:inlineVariableOutputOperand}.

\begin{lstlisting}[firstnumber=6, caption=variable-O3.s]
    main:                                   # @main
        .cfi_startproc
    # %bb.0:
                                            # kill: def $edi killed $edi def $rdi
        movl	$-10, -4(%rsp)
        #APP
        # annotation: -4(%rsp) = int
        #NO_APP
        movl	-4(%rsp), %eax
        addl	%edi, %eax
        addl	$134217635, %eax        # imm = 0x7FFFFA3
        retq
\end{lstlisting}

Further investigation revealed that the issue arises as the inline \code{asm} specifies the location of the operand must be in memory with the constraint \code{"m"(a)}. Rather than limiting the location to memory, allowing any operand available to be used allows for more optimisations to be performed by the compiler. As such, the following line of \code{asm} was tested. 

\begin{lstlisting}[language=C, numbers=none]
asm("# annotation: %0 = int" : : "X"(a))
\end{lstlisting}

Here the constraint \code{"X"(a)} specifies that any operand whatsoever is allowed. The resulting optimised assembly is as follows.

\lstinputlisting[caption=variable-O3.s]{source/inline/annotated/variable-O3.s}

Each of the different types of annotations were additionally annotated using the same method. Some interesting results occur from this annotation technique. To begin with, because all variables have been optimised away, the location of the simple variables have instead been replaced with their value. For example, line 20 shows that the value of -10 was stored in an integer. This behaviour is identical to that observed by the CompCert annotations. Additionally, an interesting scenario occurs when working with floating point numbers. Although they too have been fully optimised out, because the inline \code{asm} has required reading their values, their value has been placed within a register designated for floating point arithmetic. 

The next goal was to attempt preserving annotations from more complex variables. To begin with, a volatile, global variable will be annotated. The test file \textit{volatile.c} was experimented on.

\lstinputlisting[caption=volatile.c, language=C]{source/inline/volatile.c}

On line 4, an \code{asm} statement has been inserted, containing the same format and information as with \textit{variable.c}. However, the variable x is instead a volatile variable. The annotated assembly with full optimisation is listed below.

\lstinputlisting[caption=volatile-O3.s]{source/inline/annotated/volatile-O3.s}

Line 11 successfully shows the annotation referencing the global variable \code{x}. However, an additional move statement has been inserted on line 9. This additional move statement was quite puzzling, however, it is inserted due to the nature of the inline \code{asm}. As the compiler does not know or understand the assembly created by the programmer, additional move statements may need to be inserted by the compiler. By working backwards, the location of \code{x} can be located within memory at the location \code{x(\%rip)}. Because the constraint \code{'X'} has been provided, any operand whatsoever is permitted. As a result, the compiler chose a register as the solution to fill the annotation's requirements. As such, an unnecessary move statement has been inserted. In such a case, the programmer should observe the result and update the \code{asm} restraint to only allow memory locations. Doing so results in the preferred behaviour with only the annotation inserted into the assembly. 

This form of inline \code{asm} was used to test annotations across the remaining test files. The annotated assembly can be viewed in Appendix \ref{app:inlineAnnotatedAssembly}.

\subsection{Quality Analysis}

Although the annotated assembly does contain additional move statements, the program still behaves as expected. However, it does result in heightened difficulty to parse the true location of the variable. For a large program with many annotations, systematically identifying where each variable is stored and modifying the constraints until move statements are eliminated is time consuming and impractical. Therefore, an analysis of the program with the additional move statements preserved will be performed. One proposed method of handling this issue is to build a tool to assist in this parsing and analysis. This tool is covered in section \ref{subsec:inlineTool}.

To analyse the affect of these additional move statements inserted, big O notation will be used. From the comprehensive testing done, no compiler optimisations were removed when annotations were addeded. Thus, for this analysis, it will be assumed that no optimisations will be reverted when annotations are added. Let \(A(n)\) be the function describing the annotated assembly, and \(B(n)\) be the function describing the non-annotated assembly. Then,

\[B(n) \in \Theta(h(n))\]

As move statements are constant, any move statement is an element of \(\Theta(1)\). Thus, for \(m\) annotations added to the program, 

\[A(n, m) \in \Theta(m \cdot h(n))\]

As can be seen, in the worst case, the program runtime increases linearly with each annotation added. For time critical or concurrent programs, this is unacceptable. Ideally, a static analysis should be possible without slowing the program for each annotation added. The added move statements are only used to perform the analysis and to identify the location of variables within the program. Therefore, one should be able to compile the program with and without annotations. Assuming all optimisations are performed with annotations turned on, an analysis conducted on a program will still be valid when annotations are turned off. As such, a tool can be created to assist in this annotation and analysis process.

\subsection{Tool Assisted Annotations}
\label{subsec:inlineTool}

To allow for the program to compile without annotations, the annotations need to be stored where they cannot affect the program output. Littering the program with extended \code{asm} statements results in a difficult and tedious process of removing them once the program is ready to be compiled. Instead, annotations can be written by the programmer using inline comments. As these comments will be ignored by the compiler, the program can be compiled on any compiler supporting their C version. To reference a variable within the annotation, the variable name can be wrapped in a \code{var} keyword. An example of an annotated file can be seen below in \textit{pread.c}.

\lstinputlisting[caption=pread.c]{source/tool/pread.c}

Annotations have been listed with an inline comment beginning with \code{// annotation:}. Following the declaration of a variable, the annotation can be listed in whatever format the programmer prefers. Line 6 shows the security policy for the variable \code{z}, denoting that its security policy is always high. Whereas, on line 7, the security policy of the variable \code{x} is dependant on the value of variable \code{z}. Also shown in this example are all the necessary annotations required for a wpif analysis.

The goal of the tool is to preserve these annotations stored in these comments. The approach for this technique is to parse the annotation comments from the source file. Once parsed, these comments can be converted to extended \code{asm} calls in the source file and recompiled. The annotated and non-annotated sources can then be compared to reconstruct the location of variables.

To develop this tool, a python program \textit{annotator.py} was created. The program takes in three arguments;

\begin{itemize}
    \item the file to compile,
    \item the location for the annotated output, and
    \item the optimisation level to compile at. 
\end{itemize}

The program then creates a clone of the source file to modify. The clone is compiled to create an assembly output without annotations. This file is stored in a temporary file until it is ready to be used. The cloned source file is then transpiled. All annotations are located and transformed into extended \code{asm}. Any special characters that would break extended asm rules are appropriately escaped to allow for their preservation to assembly. This code is then injected into the cloned source file, ready for recompilation. An example of a transpiled source file can be seen in Listing \ref{lst:preadTranspiled}.

\lstinputlisting[caption=pread-transpiled.c, label=lst:preadTranspiled]{source/tool/transpiled.c}

As can be seen, each of these annotations follow the annotation style developed earlier in this section. Each reference of a \code{var} has been appropriately replaced with a variable constraint and corresponding operand. Once again, the transpiled source file is compiled to assembly, and the results of the assembly are compared against the non-annotated counterpart. If the only difference in each of the assembly outputs is the annotations, the process is complete and the program exits. However, if there is a difference, the programmer is notified and the difference of the two compiled sources are listed.

The annotator program uses the \code{clang} compiler, however, because of the decoupled nature of the program, any compiler supporting extended \code{asm} could be used by the program. 

\section{LLVM Compiler Modification}
The final technique of modifying the LLVM compiler was not experimented on. This was primarily due to two reasons. To begin with, the primary objective of this thesis is to explore techniques that do not modify the compiler, and instead work alongside the functionality of the compiler to preserve annotations. It is well known and documented that modifying the compiler to preserve annotations is possible and successful, as in the case of Vu et al. \cite{vu2020secure} Additionally, earlier success through the technique of using inline assembly allowed for more time to be allocated to exploring and improving this technique, as seen in \ref{sec:inlineAssembly}. Therefore, evaluating compiler modification for static analysis purposes was not performed in this research.
